\chapter{Implementation}
\label{chap:implementation}

\section{Overview}

In this chapter, we will the deployment of the Socrates framework in two domains, restaurant recommendation and movie tickets. We will demonstrate how the framework's robustness, modularity, and ability to be re-targeted across new domains. For each domain, we describe the setup configurations used, implementation details for the user simulator, dialog agent, and the various constituent models implementations like natural language generation and knowledge base querying. Note the full implementation code can be found in the appendix. We will focus on the strategies deployed and in particular the flexibility of the framework to handle different implementations.

\section{Restaurant Recommendations}

\subsection{Overview}

The restaurant recommendation domain focuses on developing a chat-bot service that makes restaurant recommendations. The objective of the Socrates simulator is to produce a set of simulated conversations between the user simulator and Restaurant Recommender dialog agent. The dialog agent will attempt to elicit as much information about the user's preferences and then attempt to provide a useful recommendation. In this task completion exercise, both the user and the dialog agent can take the first turn. 

As mentioned in the Design chapter, the framework is driven by configuration files. The researcher can easily modify and run different experiments through updated a human readable yaml file or a programmatically generated json file. For this use case, four configuration files are created to capture the following information:
\begin{itemize}
	\item Simulation settings: contains all the details around simulation criteria (e.g. number of rounds, first speaker, save setting, etc) as well the configuration details for dialog domain, user simulator and dialog agent. 
	\item Dialog domain: describes the dialog action space, inform and request slots, and specifies valid user goals templates.
	\item NLG templates for user simulator and agent: a rules based template to generate natural language utterances 
\end{itemize}
Additionally, we created custom modules to implement the user simulator and dialog agent for an end-to-end dialog simulation. 

We used the Dialog State Tracking Challenge 2 (DSTC2) restaurant data and service as a model for our use case. The DSTC2 is research challenge put together by the University of Oxford and Microsoft to advance dialog research. For DSTC2, the goal was track the state of multi-stage conversations between real humans and a expert restaurant recommender service. The restaurant recommender service would provide recommendations based off the following user preferences: cuisine, area, and price range. DSTC2 also provided a rich and deep set of training data that allowed to model both neural network based approaches for NLG and NLU, as well as rule based approaches. 

Below we will describe our strategy for developing the user simulator and dialog agent using the Socrates Simulator Framework and implementation choices for their constituent parts. 

\subsection{Domain}

\begin{figure}[h!]
	\caption{ Restaurant Dialog Domain}
	\label{fig:restaurant_domain}
	\begin{lstlisting}
	# Dialog Action Space
	dialog_acts: [ inform, confirm, affirm, request, 
			negate, greetings, bye]
	
	# Request Slots
	request_slots: [ address, area, cuisine, 
			 phone, pricerange, postcode, name ]
	
	# Inform slots
	inform_slots: [ cuisine, pricerange, name, area ]
	
	# Valid User Goals Temaplates
	valid_user_goals:
	- [ name ]                    # User wants restaurant name
	- [ name, address ]           # User wants name and address
	- [ name, address, phone ]    # User wants name, address, and phone
	- [ name, phone ]             # Use wants name and phone
	\end{lstlisting}
\end{figure}
The first thing we setup is the dialog domain configuration file. This file is used to generate a domain object that will be used by both the user simulator and the dialog agent. For restaurant recommendations, we used dialog domain described in the DSTC2 ontology. The figure above details the s dialog action space, and the different inform and request slot types. The dialog domain was expressed as a yaml file, where each key captured the salient attributes of the domain. 

One important feature of this file is the valid user goals templates section. It specifies the valid types of goals a user may have when engaging the dialog agent. The user simulator will use it generate random goals that can explore the preference space and test the robustness of the agent. 

Note, the figure above does not include the all the possible inform and request slot values. Please see the [appendix ref] for the full configuration file.  

\subsection{Domain Knowledge Base }

The knowledge base for this use case is csv with various restaurants from the Cambridge area. Each row represents a unique restaurant and the column values capture the restaurant's cuisine, price range, area, phone number, and address. The data was scraped by Yelp. For the scraping script, see the appendix. The restaurants list was loaded as DomainKBtable object. The DomainKBtable loads the csv files as in memory pandas dataframes and provides a a set of methods for querying. 

\subsection{Natural Language Understanding Implementations}

The goal of NLU implementation is to parse a natural language utterance into a DialogAction object. The parsed dialog action contains the intent of the utterance, i.e. the dialog act, and any entity / entity types, i.e. the dialog parameters, which are contained in the utterance. A rules based model and neural model were implemented to demonstrate the different models a researcher can use to support the natural language understanding. 

\subsubsection{Rules Based NLU}

For the rules based approach, our parser has two part strategy. The first is to classify the intent of the utterance and map it to a dialog act. The second to is run an entity extraction pass and attempt to extract the entities contained in the utterance and map them to inform slot types. 

The algorithm for the intent classification consists of two parts. The first is running the utterance through a question classifier ( implemented from \cite{chewning_lord_yarvis_2015} ). If the utterance is a question, we classify it as a request dialog act. Otherwise run though a set of regular expression matches and return the corresponding dialog acts ( see \ref{fig:intent_clf} ). Given the wide range of potential string matches for inform, we use that as the default dialog act. 

\begin{figure}[h!]
	\caption{ Intent Classifier Algorithm}
	\label{fig:intent_clf}
	\begin{lstlisting}
	Classify Intent
		input: natural languge utterance
		
		IF input is a question:
			return 'request'
		ELSE IF input contains [you, you want, right?]:
			return 'confirm'
		ELSE IF input contains [yes, yeah, yup, correct, right]]:
			return 'affirm'
		ELSE IF input contains [no, nope, wrong, incorrect]:
			return 'negate'
		ELSE IF input contains [hi, hello]:
			return 'greetings'
		ELSE IF input contains [bye, goodbye, thanks, thank you]:
			return 'bye'
		ELSE
			return "inform"
	\end{lstlisting}
\end{figure}

After the intent is classified, we then check to see if any entities are found in the utterance that can be mapped to entity types. To achieve this, we first lookup all the slot values defined in the domain object. Next, we create a reverse map dictionary, where each unique slot value ( i.e. the entity ) is mapped to a slot type (e.g. cuisine or price range ). The NLU parser then tokenizes the utterance into set a of word trigrams and check if a slot type exists for the token in the reverse map. All positive matches are added to the dialog params list in the DialogAction object. Finally, the parse utterance will return a DialogAction object with parsed dialog act and a list of dialog parameters. 

\subsubsection{Neural Model for NLU}

We also implemented a simple neural machine translation model for our NLU module. 

The DSTC2 provides a large labeled dataset of conversations between human users and an human expert posing as the dialog agent. Between the train and development set were about 2000 annotated calls. All speech utterances ( between the human and agent ) were parsed and annotated. To train the NMT model, we extracted out all utterances and the corresponding parses ( expressed as json string objects ). The model was trained over X epochs and released.

SERT NMT model diagram.



\subsection{Natural Language Generation Implementations}

The objective of the NLG module is to generate a natural language utterance, provided a dialog action. For the restaurant recommendation use case, we implemented both a template based model and a neural model. 

\subsubsection{Template  Based Model}

The template model is defined by a yaml file (see figure \ref{fig:res_nlg}). The nlg template is loaded into memory as a nested python dictionary. The first layer of keys are indexed by the dialog acts ( e.g. request, inform, etc), and the corresponding values are dictionaries indexed by specific slot types. In the case where there are the are no slot types ( e.g. affirm ), the default value is used. The natural language templates are stored in lists at the values for the slot types.

\begin{figure}[h!]
	\caption{ Example NLG template for User Simulator }
	\label{fig:res_nlg}
	\begin{lstlisting}
	 affirm:
		default: [ "Yes.", "Yup.", "Yes, that's right." ]
	 greetings:
		default: [ "Hi, I'm looking for a restaurant.",
			   "Hi! Can you help me find a restaurant?" ]
	 inform:
		cuisine: [ "I'd like find a restaurant that serves $CUISINE.",
			   "I'm looking for $CUISINE food.",
			   "I want to eat $CUISINE food." ]
		pricerange: [ "I'm looking for a $PRICERANGE priced restaurant.",
			      "Looking $PRICERANGE priced food." ]	
	\end{lstlisting}
\end{figure}

The logic then is straight forward to generating a natural language utterance. The get utterance method in simulator will be passed a dialog action object which contains the dialog act and a list dialog parameters. We first look up the dialog act in the nlg template dictionary. Next we look up the slot values ( if any ) for the specific language templates. The slot values are passed in the dialog params property of the DialogAction object. Additionally, language templates that have multiples slot types, are indexed by combination of the slot types into a single string. We take the slot types, lower case them, arrange them by alphabetical order, and concatenate them together with comma separator into a single string. For example, "I want \$PRICE \$CUISINE" would be indexed by the string "cuisine,price". Finally, we randomly sample the list from the list of potential language template, substitute slot values, and return a generated natural language utterance. 


\subsection{Neural Model for NLU}

We also implemented a simple neural machine translation model for our NLU module. 

The DSTC2 provides a large labeled dataset of conversations between human users and an human expert posing as the dialog agent. Between the train and development set were about 2000 annotated calls. All speech utterances ( between the human and agent ) were parsed and annotated. To train the NMT model, we extracted out all utterances and the corresponding parses ( expressed as json string objects ). The model was trained over X epochs and released.

SERT NMT model diagram.


\subsection{User Simulator}

We designed a rule based user simulator given the simplicity of the dialog domain. In this use case, the user has a set of hidden preferences and is looking get name of a restaurant that satisfies those preferences from the dialog agent. Additionally, the user may also want to get some the restaurant's phone number and address.

At the start of each dialog round, the user simulator will be provided a goal from the Dialog Manager. For demonstration purposes we defined both an explicit set of user goals  and valid goal template for random goal generation The figure below shows what a sample goal would look like. 

\begin{figure}[h!]
	\caption{ Example User Goal. User is seeking the name and phone number of a cheap Chinese restaurant}
	\label{fig:ex_user_goal}
	\begin{lstlisting}
	inform_slots:
		cuisine: "chinese"
		pricerange: "cheap"
	request_slots:
		name: "UNK"
		phone: "UNK"	
	\end{lstlisting}
\end{figure}

The rules simulator we designed is a separate python module that will be dynamically loaded by the Dialog Manager at simulation time. The rules simulator class inherits the base UserSimulator class, which in turn is a subclass of Speaker. The rules simulator will implement two key public methods defined by the Speaker, \textit{next} and \textit{get utterance}. From the dialog managers point of view, what the rule simulator does under the hood is completely hidden. The \textit{next} method takes in the opposing speakers speech utterance 

The user agenda, which captures what the user simulator will communicate to the dialog agent, is simply the concatenation of the inform and request slot lists stored in the user goal. Functionally the user agenda is a stack of pending dialog acts that user will say over the course of the conversation. All key-value pairs captured in the corresponding inform and request slot lists are mapped to inform and request dialog acts. Over the course of the dialog, the top item at the stack, which gets popped, contains that dialog action for what the user simulator will do next. That action would be passed the the simulator's internal natural language generation module to generate a natural language utterance. 

 For memory efficiency, we do not actually implement the agenda, as the information already exists in user goal. Instead we pop directly from the inform slots list or defer the action generated by the next method for responses to the dialog agent. We keep track of state of conversation by check how many of the request slots have been filled with real values. The rules simulator runs sequentially through the request slots at the end of each conversation and updates its internal DialogStatus enum object. The logic for how the user simulator responds to incoming speech acts from the dialog agent is handled by the \textit{next} method.
 
 \begin{figure}[h!]
 	\caption{ Internal logic for the rules simulator. }
 	\label{fig:ex_user_goal}
 	\begin{lstlisting}
 	response_router = { "greetings": respond_general,
 	"inform": respond_to_suggestion,
 	"random_inform": respond_random_inform,
 	"request": respond_request,
 	"confirm": respond_confirm,
 	"bye": respond_general}
 	
 	\end{lstlisting}
 \end{figure}

The \textit{next} is the driver of the rules simulator. It first will first attempt to parse the agents incoming dialog action and then respond to it using an internal dispatch tree. The internal logic of the \textit{next} method is a simple dispatch dictionary, where incoming dialog acts are mapped to resolver functions. Each response function has the same method signature, which is to take in a DialogAction object and return back a DialogAction that represents the user's response to dialog agent. The dialog agent's dialog action space is limited. The agent will respond with one of the following dialog acts:
\begin{itemize}
	\item greetings: the agent will greet the user and list it's services
	\item request: the agent will ask a probing question to elicit the users preferences
	\item inform: the agent will supply the user with information (usually tied to the user request request slots)
	\item confirm: the agent will ask the user to confirm if it understood the user's intent 
	\item bye: the agent will end the conversation  
\end{itemize}

The implementation of how the resolvers for the greetings, bye, and confirm responses are straight forward. The code implementation can be found in the appendix. For greetings resolver, if the rules simulator is the first speaker, it will invoke the random inform method one or several inform slots, which will be translated into an inform speech act. Otherwise, it handle the agent dialog action with the dispatch dictionary. 

The resolver for responding to the agents request actions is a bit more involved. The simulator will looked at the parsed request slots types the agent is asking about and attempt to find corresponding inform slots in it goal object. For example, if the dialog agent asks "What cuisine do you prefer?", the simulator will look up cuisine in its internal goal and return the answer Chinese (i.e based on the goal in \ref{fig:ex_user_goal}). In the case where requested slot type is not found in the user's inform preferences, the simulator will respond with either "I don't know" or "I don't care". This null response can be configured in the simulation configuration file, where the response either set to one of those two options or randomly chosen. Additionally, to simulate a realistic user, at configuration time, the researcher can also set the probability with which the simulator will "lie" or "change its mind" about its preferences. In those cases, the simulator will randomly sample the provided slot values in the dialog domain object and return a different slot value. 

If rules simulator has exhausted the informing the dialog agent of all its preferences, the simulator will then pop values from its request slots list and ask for a recommendation. If the dialog agent sent an inform action, the inform resolver method would update the request slot with new provided information. So for example, if the dialog agent made the suggestion, "Check out Golden Dynasty", the "UNK" value in \ref{fig:ex_user_goal} would be replaced by "Golden Dynasty". If all the "UNK" values in the request slots were filled with real values, the user simulator would update its internal status to complete and issue the bye action. 

As described above, the nlu and nlg models were set by the researcher in the simulation configuration file. By inheriting the UserSimulator class, the get utterance and parse utterance methods are also inherited and available to the researcher. Both method essentially call the corresponding methods in nlg and nlu objects. 

\subsection{Dialog Agent}

The restaurant agent was developed to illustrate how to incorporate an external dialog agent into the simulation framework. Since we do not have an existing restaurant recommendation agent, we developed a simple rule based agent. The goal of the agent is to capture all the user's preferences and then make a suggestion from its knowledge base of Cambridge restaurants. Like the user simulator, the public facing methods the dialog manager interacts with are \textit{next} and \textit{get utterance}. 

For the restaurant agent, we follow a simple rules approach. The agent expects to interact with the following dialog acts: greetings, affirm, negate, request, inform, bye. In situations where the agent encounters an unknown dialog act, it will repeat its last dialog act. At the beginning of conversation round, dialog agent internally resets its goal ( \ref{fig:ex_res_agent_goal} ). 

  \begin{figure}[h!]
 	\caption{ Restaurant Agent Goal }
 	\label{fig:ex_res_agent_goal}
 	\begin{lstlisting}
 	inform_slots: None
 	request_slots:
 		cuisine: UNK
 		area: UNK
 		pricerange: UNK 	
 	\end{lstlisting}
 \end{figure}

If restaurant agent goes first, it issues a greetings action. If the agent is not responding to user, it will sequentially pop one item from it request slots and issue a request dialog act. Once the restaurant agent has collected information from the user, it will attempt to make a suggestion from the knowledge base stored in the domain object. This is accomplished by calling the get suggestion method provided by the domain object.

\section{Movie Tickets}

\subsection{Overview}
The goal of the Movie Booking agent is help the end user purchase movie tickets. A similar domain agent was developed for the \cite{li_usersim} paper. Unfortunately, I was unable to use their agent and domain knowledge base in the context of the Socrates User Simulator. Much of their data was crowd sourced from Amazon Mechanical Turk and their use cases were a bit convoluted. There was a confusing overlap between the dialog action space for the user and agent which resulted in the user simulator making unrealistic utterances and dialog acts. There was a high ratio of nose and data quality issues which would have made it difficult to replicate.  

Drawing inspiration from their use case, the agent we developed is a bit more refined and modeled after something you would see on a movie booking site like Fandango. The agent will help identify movies for the user based on user preferences and also collect the user's payment details if the user choses to book a ticket. As the goal for this implementation is mainly to demonstrate the end to end dialog simulation framework and the user simulator, we did not implement an actual reservation database, a payment processing system, and a large move show times catalog. There are stubs for where the dialog agent would theoretically consult external resources in the aiding the user. We focus below on the the implementation details for the user simulator and domain modeling exercise.  

\subsection{Domain}

 \begin{figure}[h!]
	\caption{ Restaurant Agent Goal }
	\label{fig:ex_res_agent_goal}
	\begin{lstlisting}
	# Domain Information
	domain_name: movie
	version: 1.0
	
	# Dialog Acts
	dialog_acts: [ inform, confirm, affirm, request, negate, greeting, bye ]
	
	# Inform slots
	inform_slots: [ city, date, movie, rating, genre, 
	theater, times, zip, no_tickets ]
	
	# Request Slots
	request_slots: [ address, city, date, movie, rating, stars,
	state, theater, times, zip, genre, no_tickets, showtime ]
	
	# Required slots
	required_inform_slots: [ no_tickets, date, cc_number, cc_exp,
	cc_zip, cc_type, ccv ]
	
	# Valid User Goals
	valid_user_goals:
	- [ tickets_booked, movie, theater, showtime, address ]
	- [ tickets_booked, movie, showtime ]
	
	# Random integer contraints
	RAND_INTEGER_RANGE: [ 1, 10 ]
	\end{lstlisting}
\end{figure}

For the movie domain, we first set up the domain configuration file. This file is used to generate the domain object that will be used by both the user simulator and dialog agent. The inform and request slots reflect information that will be communicated and captured specifically for the movie book use case. The user will haves preferences that include desired movie genre, minimum movie rating, show times, and theater location. From the agent's point of view, it will need to elicit the number of tickets the user wants to buy, the movie's name, showtime, and theater location. Additionally, the agent will collect the user's credit card information to complete the "purchase" of the tickets. 

The  domain configuration file is intentionally flexible. Compared to the the configuration file for the restaurant domain, you will notice two new sections, required slots and random integer range. The required slots indicate the inform slots the user goal must have in its inform slots. Here, that is the credit card information, which the agent uses to charge the user for the purchase of the tickets. The random integer range is used to generate at random, the number of tickets the user simulator wishes to purchase. 

The configuration file gets read into Socrates Sim as a python dictionary and then converted into a domain object. Standard domain information (dialog acts, inform and request slots, valid goals) are mapped directly as object properties. Additional information is stored in a dictionary called \textit{additional\_params} and is also part of the domain object. This design allows the user flexibility in defining the domain while maintaining a minimum level of standardization. 

\subsection{Domain Knowledge Base}

We generated a sample movie database using Fandango movie showing data. To acquire the data, we developed a simple web scraper that used BeautifulSoup and the python requests library. The scraper navigated to Fandango.com, and extracted movie show times and move information. Since the data is only meant to be a proof of concept, we limited scraping to a single location (movie theaters with 2 miles of Cambridge, MA) and showings for one week. As result we scraped show times for about 9 movies across 6 theaters offered and all possible show times. Unfortunately, Fandango does not make available genre information and star ratings about the movies on the website. I randomly assigned each unique movie genre from 6 genres (action, romance, comedy, adventure, thriller and horror ) and star rating score from 1 to 5.

The scraped data was stored as in a csv file. Like the restaurant domain, I created a DomainKBtable object, which loads in memory the csv file as a pandas data frame. The DomainKBtable provides several methods for querying and sampling.

\subsection{ Natural Language Understanding}

For the movie domain, there were no publicly available datasets to seed a neural based nlu model. We reused the rule based nlu model developed for the restaurant domain. No new code had to be written. The rule based model ingests a domain object and uses the domain's inform and request slot values for parsing. It should be noted that the rule based nlu is brittle as it will use the literal slot values for reverse matching the entity. If the user simulator generates paraphrase or alters the spelling of the entity, the model will fail to map the entity to he appropriate slot type. For demonstration purposes that is alright, as the goal of this use case is demonstrate re-targetability and flexibility of the framework.

No new code had to be written to incorporate the a new nlu model for the movie domain, demonstrating the value of creating a higher level NLU abstract class. If researcher wishes to build a more robust and finely tuned nlu model, they can easily plug it into the framework. This is accomplished by writing a simple wrapper that inherits the NLU abstract class and 
 specifying the new location of this new nlu object in the simulator configuration file.

\subsection{ Natural Language Generation}

We used a template based model for generating the user simulator utterances. As mentioned above, there was no publicly available data to seed the neural model. The language templates were written as yaml file.

\subsection{User Simulator}



\subsection{Dialog Agent}



%%% Local Variables: 
%%% mode: latex
%%% TeX-master: "main"
%%% End: 
