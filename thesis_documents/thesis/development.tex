\chapter{Development}
\label{chap:dev}

This chapter discusses the tools and methodologies employed in the code development of this system. 

\section{Development  Language}
%The framework was written in Python 3.7, which at the time of submission is the latest python version. It is worth noting that the research implementation of the user simulator described in \cite{li_usersim} was written in Python 2. Python 3x is the preferred version for production python products. Most major data science and machine learning research python libraries no longer support Python 2. Python 3 provides many useful features and performance upgrade that make writing and deploying python projects more efficient and effective. In addition to updated syntax that allows for more expressive coding, the standard library was extended to support new data types (ordered dictionaries, enumerated types, data classes). Additionally, Python 3.5 introduced type annotation, which allows for the writing of cleaner, better documented, and unambiguous code. We describe type annotations further below.

The framework was written to adhere to PEP8 standard and all method signatures have type annotation. The hope is that good software documentation and standard coding styles will allow future contributers to easily debug, modify, and build new modules for the framework. 

\section{Unit Testing}


\section{Development  Tools}
\label{sec:devtools}



%%% Local Variables: 
%%% mode: latex
%%% TeX-master: "main"
%%% End: 
