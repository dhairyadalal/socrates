%% This file contains personal customizations like graphics, math,
%% algorithm/code listing, and glossary.


% ------------------------------------------------------------
% graphics

\usepackage{graphicx}
\usepackage{epsf}
\usepackage{multicol}
\usepackage{graphics,color,latexsym}



%% ------------------------------------------------------------
%% basic math packages

\usepackage{amsmath}
\usepackage{amssymb}
\usepackage{amsthm}
\usepackage{amsfonts}
\usepackage{theorem}

% other math packages
\usepackage{bm}

% for diagrams
\usepackage{amscd}

% for conditionals
\usepackage{ifthen}

% linear
%\usepackage{esvect}

\usepackage{enumerate}

% change the way items are numbered in the enumerate environment
\renewcommand{\labelenumi}{(\alph{enumi})}

%% ------------------------------------------------------------
%% some basic math definitions


%% ------------------------------------------------------------
%% some math environments


\newlistof{myequations}{equ}{\listequationsname}
\newcommand{\addmyequations}[1]{
  \addcontentsline{equ}{myequations}{\protect\numberline{\theequation}#1}\par
}
\cftsetindents{myequations}{1.5em}{3em}


%% ------------------------------------------------------------
%% some math bold faces



%% ------------------------------------------------------------
%% for some CS

%\def\O{\mathcal{O}}

\newcommand{\sym}[1]{\texttt{#1}}
\newcommand{\algor}[1]{\textsf{\textsc{#1}}}


% ------------------------------------------------------------
% for code and algorithms 

\usepackage{verbatim}
\usepackage[chapter]{algorithm}
\usepackage{listings}
%\usepackage{algorithmic}
\usepackage{afterpage}		% help with multi-page listing



\newcommand{\INPUT}{\textbf{Input}}
\newcommand{\as}{\ensuremath{\leftarrow}}

% these should be after hyperref because of conflicts
% this might help
%\newcommand{\theHalgorithm}{\arabic{algorithm}}

% for lsting
%\renewcommand{\lstlistingname}{\textbf{Algorithm}}
%\renewcommand{\lstlistlistingname}{Algorithms}


\lstdefinelanguage{Algorithm}%
{morekeywords={input,for,each,while,if,else,then,do,to,break,end,print,return,function},%
  aboveskip=\smallskipamount,
  belowskip=\smallskipamount,
  xleftmargin=25pt,         
  basicstyle=\footnotesize,	% \tiny, \scriptsize, \small, \footnotesize, \bfseries, \upshape
  sensitive,%
  mathescape,%
  commentstyle=\footnotesize\upshape,%
%  morecomment=[s]{\{}{\}},%
  morecomment=[s]{\{}{\}},%
  %morecomment=[l]\%,%
  %morestring=[m]'%  
%  fontadjust=true,		%
  texcl=true,			% include TEX source
  numbers=left,			% where to put the line-numbers
  numberstyle=\tiny,		% 
  numbersep=10pt,		% how far the line-numbers are from the code
  captionpos=b,		        % sets the caption-position to bottom
 }


\lstdefinelanguage{JavaText}%
{
  aboveskip=\smallskipamount,
  belowskip=\smallskipamount,
  xleftmargin=25pt,         
  basicstyle=\footnotesize\ttfamily,	% \tiny, \scriptsize, \small, \footnotesize, \bfseries, \upshape
  frame=single,			% adds a frame around the code ('single', 'b')
  numbers=left,			% where to put the line-numbers
  numberstyle=\tiny,		% 
  numbersep=10pt,		% how far the line-numbers are from the code
  captionpos=b,		        % sets the caption-position to bottom
}
\lstdefinelanguage{Java}%
{
}


\lstdefinelanguage{XMLText}%
{
  aboveskip=\smallskipamount,
  belowskip=\smallskipamount,
  xleftmargin=25pt,         
  basicstyle=\footnotesize\ttfamily,	% \tiny, \scriptsize, \small, \footnotesize, \bfseries, \upshape
  frame=single,			% adds a frame around the code ('single', 'b')
  numbers=left,			% where to put the line-numbers
  numberstyle=\tiny,		% 
  numbersep=10pt,		% how far the line-numbers are from the code
  captionpos=b,		        % sets the caption-position to bottom
}
\lstdefinelanguage{XML}%
{
  basicstyle=\footnotesize\ttfamily,	% \tiny, \scriptsize, \small, \footnotesize, \bfseries, \upshape
}



%% The default listing format
%% 
\lstset{ %
  %language=Java,		% default language, set with \lstset{language=XXX}
  %------- MARGINS and FRAMES --------
%  aboveskip=\smallskipamount,
%  belowskip=\smallskipamount,
  xleftmargin=0pt,         
  % xleftmargin=17pt,
  %framexleftmargin=17pt,
  %framexrightmargin=5pt,
  %framexbottommargin=4pt,
  % frame=single,			% adds a frame around the code ('single', 'b')
  % float=true,			% make it a float
  %------- FONTS --------
  basicstyle=\footnotesize\ttfamily,	% \tiny, \scriptsize, \small, \footnotesize, \bfseries, \upshape
  % keywordstyle=\bfseries,	% \color{red},
  % commentstyle=\upshape,	%         
  % stringstyle=\color{white}\ttfamily,	%
  %------- BACKGROUND COLOR --------
  %backgroundcolor=\color{white},  % choose the background color. Must add \usepackage{color}
  %backgroundcolor=\color{lightgray},
  %------- HORIZONTAL SPACING  --------
  % fontadjust=true,		%
  keepspaces=true,		%
  showspaces=false,		% show spaces adding particular underscores
  showstringspaces=false,	% underline spaces within strings
  showtabs=false,               % show tabs within strings adding particular underscores
  % tabsize=3,			% sets default tabsize to 2 spaces
  columns=[l]flexible,		% 'flexible', 'fixed'
  % texcl=true,			% include TEX source
  %------- NUMBERING --------
  numbers=none,			% where to put the line-numbers
  %numberstyle=\tiny,		% 
  %numbersep=10pt,		% how far the line-numbers are from the code
  % stepnumber=2,		% the step between two line-numbers. If it's 1 each line will be numbered
  %------- CAPTION --------
  % caption=\lstname		% default caption name
  %captionpos=b,		% sets the caption-position to bottom
  %------- LINE BREAKING --------
  breaklines=true,		% sets automatic line breaking
  % breakatwhitespace=false,    % sets if automatic breaks should only happen at whitespace
  %------- COMMENTS --------
  % escapeinside={\%*}{*)}          % if you want to add a comment within your code
}

%\cftsetindents{algorithm}{1.5em}{3em}

%\renewcommand{\lstlistingname}{Listing \arabic{chapter}.\arabic{section}}




% ------------------------------------------------------------
% for url
\usepackage{url}

\usepackage[pdftex,colorlinks,plainpages=false,pdfpagelabels]{hyperref}
%\usepackage[ps2pdf=true,colorlinks]{hyperref}
%\usepackage{hyperref}	%% this should be last

\usepackage[figure,table]{hypcap} % Correct a problem with hyperref
\hypersetup{
% these are from beamer
  bookmarks=true,%
%  bookmarksopen=true,%
  pdfborder={0 0 0},%
  pdfhighlight={/N},%
  linkbordercolor={.5 .5 .5},
%%
  bookmarksnumbered,
  pdfstartview={FitH},
  citecolor={black},
  linkcolor={black},
  urlcolor={black},
  pdfpagemode={UseOutlines}
} 
\makeatletter
\newcommand\org@hypertarget{}
\let\org@hypertarget\hypertarget
\renewcommand\hypertarget[2]{%
  \Hy@raisedlink{\org@hypertarget{#1}{}}#2%
} \makeatother 

% ------------------------------------------------------------
% Glossary

\usepackage{etex}   % to avoid: 'supp-mis.tex:197: No room for a new \dimen'
%\usepackage[nonumberlist,section=chapter,numberedsection=autolabel]{glossaries}
\usepackage[section=chapter,numberedsection=autolabel]{glossaries}

%% make glossary
% latex
% makeindex -s main.ist -t main.glg -o main.gls main.glo
% ./makeglossaries.pl main
% latex; latex

% for glossary
\makeglossaries
\loadglsentries{glossary}

% load all glossary entries
%\glsaddall

%A \gls{sample} entry and \gls{aca}. Second use: \gls{aca}.
%Plurals: \glspl{sample}. Reset acronym\glsreset{aca}.
%First use: \glspl{aca}. Second use: \glspl{aca}.


%% ------------------------------------------------------------
%% OTHER CUSTOMIZATIONS

% for landscape pages
\usepackage{lscape}

\DeclareGraphicsExtensions{.pdf,.eps,.jpg,.mps,.png}

\usepackage{pifont}
